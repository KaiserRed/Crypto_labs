\documentclass[12pt]{article}

\usepackage{fullpage}
\usepackage{multicol,multirow}
\usepackage{tabularx}
\usepackage{ulem}
\usepackage[utf8]{inputenc}
\usepackage[russian]{babel}
\usepackage{listings}
\usepackage{hyperref}
\usepackage{graphicx}
 \usepackage{amsmath}
\DeclareGraphicsExtensions{.png}


\begin{document}

\section*{Лабораторная работа №\,3 по курсу криптографии}

Выполнил студент группы М8О-308Б \textit{Галкин Алексей Дмитриевич}.

\subsection*{Условие}
Сравнить:
\begin{enumerate}
\item Два осмысленных текста на естественном языке.
\item Осмысленный текст и текст из случайных букв.
\item Осмысленный текст и текст из случайных слов.
\item Два текста из случайных букв.
\item Два текста из случайных слов.
\end{enumerate}

Считать процент совпадения букв в сравниваемых текстах – получить дробное значение от 0 до 1 как результат деления количества совпадений на общее число букв. Расписать подробно в отчёте алгоритм сравнения и приложить сравниваемые тексты в отчёте хотя бы для одного запуска по всем пяти случаям. Осознать, какие значения получаются в этих пяти случаях. Привести соображения о том, почему так происходит. Длина сравниваемых текстов должна совпадать. Привести соображения о том, какой длины текста должно быть достаточно для корректного сравнения.


\subsection{Основные понятия}

Частота совпадений символов — вероятность того, что два символа на одинаковых позициях в разных текстах совпадут:

\begin{equation}
    P_{\text{совпад}} = \frac{\text{Число совпавших символов}}{\text{Общая длина текста}}
\end{equation}

\textbf{Особенности:}
\begin{itemize}
    \item Естественный язык обладает избыточностью (неравномерное распределение символов)
    \item Теоретическая вероятность совпадения для случайных английских букв: $\frac{1}{26} \approx 3.85\%$
    \item Слова из случайных букв ближе к естественному языку, чем полностью случайные символы
\end{itemize}

\subsection{Применение в криптографии}
\begin{itemize}
    \item Оценка стойкости шифров
    \item Обнаружение стеганографии
    \item Генерация криптографических ключей
    \item Анализ случайности ГПСЧ
\end{itemize}


\subsection*{Метод решения}
\parВ качестве осмысленных текстов на естественном языке были взяты <<Государь>> Н. Макиавелли и <<Преступление и наказание>> Ф. М. Достоевского, оба на английском языке.\\
Ссылки:\\
https://www.gutenberg.org/files/1232/1232-0.txt\\
https://www.gutenberg.org/files/2554/2554-0.txt\\

\parТекст из случайных слов генерируется из следующего словаря (чуть больше 466 тысяч английских слов):\\
https://raw.githubusercontent.com/dwyl/english-words/master/words.txt\\

\par Текст из случайных букв генерируется из букв английского алфавита в обоих регистрах и состоит из слов длиной от 3 до 10 знаков.

\par Алгоритм сравнения: параллельно обходим оба текста, сравниваем знаки на одинаковых позициях. Если знаки совпадают, то увеличиваем счётчик совпавших символов на 1. Потом вычисляем процент совпадения. Сравнение регистрозависимое.

%\includegraphics[width=\linewidth]{1.png}\\


\subsection*{Результат работы программы}
\begin{lstlisting}
(.venv) alexey@alexey-Yoga-Slim-7-Pro-14IHU5:~/code/crypta/lab_3$ python3 main.py 
Case #1: two meaningful texts in natural language.
Text length: 287535
Match: 0.06482341280191976
Case #2: meaningful text and text from random letters.
Text length: 287535
Match: 0.03433599387900603
Case #3: meaningful text and text from random words.
Text length: 287535
Match: 0.05737110264837325
Case #4: two texts from random letters.
Text length: 1000000
Match: 0.0322717
Case #5: two texts from random words.
Text length: 1000000
Match: 0.05785960000000001
\end{lstlisting}

\subsection*{Анализ результатов}

\textbf{Case 1-2:} Разница между 6.48\% и 3.43\% объясняется избыточностью естественного языка. Частые буквы (e, t, a) дают больше совпадений.

\textbf{Case 3:} 5.74\% для случайных слов подтверждает, что даже случайная комбинация реальных слов сохраняет языковые паттерны.

\textbf{Case 4-5:} Стабильные результаты для больших текстов (3.23\% и 5.79\%) показывают, что 1 млн символов достаточно для устойчивой статистики. 
\subsection*{Листинг программного кода}
\begin{lstlisting}[language=Python]
    import random
    import string
    import getopt
    import os
    import sys
    
    import urllib.request
    
    
    CNT_RANDOM_TEXTS = 10
    LEN_RANDOM_TEXT = 10 ** 6
    CASES = 5
    
    
    USAGE = """
    Syntax: main.py [--cases=#]
    
      Flags:
        cases=#
            Numbers of cases to use. By default all cases are used.
    
            1 -- two meaningful texts in natural language
            2 -- meaningful text and text from random letters
            3 -- meaningful text and text from random words
            4 -- two texts from random letters
            5 -- two texts from random words
    
        Example:
            --cases=1,3
    
    """
    
    
    def count_common_letters(text1, text2):
        cnt = 0
        for char1, char2 in zip(text1, text2):
            if char1 == char2:
                cnt += 1
        return cnt
    
    
    def match_perc(text1, text2):
        return count_common_letters(text1, text2) / len(text1)
    
    
    def gen_random_letters(n):
        text = ''
        while len(text) < n:
            len_word = random.randint(3, 10)
            word = ''.join(random.choice(string.ascii_letters) for _ in range(len_word))
            text += ' ' + word
        rem = len(text) - n
        if rem != 0:
            text = text[:-rem]
        return text
    
    
    def gen_random_words(n):
        # Alternative word list source
        url = 'https://raw.githubusercontent.com/dwyl/english-words/master/words.txt'
        response = urllib.request.urlopen(url)
        words = response.read().decode().splitlines()
        text = ''
        while len(text) < n:
            text += ' ' + random.choice(words)
        rem = len(text) - n
        if rem != 0:
            text = text[:-rem]
        return text
    
    
    def case1():
        print("Case #1: two meaningful texts in natural language.")
        url = 'https://www.gutenberg.org/files/1232/1232-0.txt'
        url2 = 'https://www.gutenberg.org/files/2554/2554-0.txt'
        response = urllib.request.urlopen(url)
        text1 = response.read().decode()
        response = urllib.request.urlopen(url2)
        text2 = response.read().decode()
        min_len = min(len(text1), len(text2))
        text1 = text1[:min_len]
        text2 = text2[:min_len]
        print("Text length: {0}".format(min_len))
        print("Match: {0}".format(match_perc(text1, text2)))
    
    
    def case2():
        print("Case #2: meaningful text and text from random letters.")
        url = 'https://www.gutenberg.org/files/1232/1232-0.txt'
        response = urllib.request.urlopen(url)
        text1 = response.read().decode()
        s = 0
        for i in range(CNT_RANDOM_TEXTS):
            text2 = gen_random_letters(len(text1))
            with open('./tests/case2_text_{0}'.format(i + 1), 'w') as f:
                f.write(text2)
            s += match_perc(text1, text2)
        s /= CNT_RANDOM_TEXTS
        print("Text length: {0}".format(len(text1)))
        print("Match: {0}".format(s))
    
    
    def case3():
        print("Case #3: meaningful text and text from random words.")
        url = 'https://www.gutenberg.org/files/1232/1232-0.txt'
        response = urllib.request.urlopen(url)
        text1 = response.read().decode()
        s = 0
        for i in range(CNT_RANDOM_TEXTS):
            text2 = gen_random_words(len(text1))
            with open('./tests/case3_text_{0}'.format(i + 1), 'w') as f:
                f.write(text2)
            s += match_perc(text1, text2)
        s /= CNT_RANDOM_TEXTS
        print("Text length: {0}".format(len(text1)))
        print("Match: {0}".format(s))
    
    
    def case4():
        print("Case #4: two texts from random letters.")
        s = 0
        for i in range(CNT_RANDOM_TEXTS):
            text1 = gen_random_letters(LEN_RANDOM_TEXT)
            with open('./tests/case4_text1_{0}'.format(i + 1), 'w') as f:
                f.write(text1)
            text2 = gen_random_letters(LEN_RANDOM_TEXT)
            with open('./tests/case4_text2_{0}'.format(i + 1), 'w') as f:
                f.write(text2)
            s += match_perc(text1, text2)
        s /= CNT_RANDOM_TEXTS
        print("Text length: {0}".format(LEN_RANDOM_TEXT))
        print("Match: {0}".format(s))
    
    
    def case5():
        print("Case #5: two texts from random words.")
        s = 0
        for i in range(CNT_RANDOM_TEXTS):
            text1 = gen_random_words(LEN_RANDOM_TEXT)
            with open('./tests/case5_text1_{0}'.format(i + 1), 'w') as f:
                f.write(text1)
            text2 = gen_random_words(LEN_RANDOM_TEXT)
            with open('./tests/case5_text2_{0}'.format(i + 1), 'w') as f:
                f.write(text2)
            s += match_perc(text1, text2)
        s /= CNT_RANDOM_TEXTS
        print("Text length: {0}".format(LEN_RANDOM_TEXT))
        print("Match: {0}".format(s))
    
    
    def print_usage(message):
        print(USAGE)
        if message:
            sys.exit('\nFATAL ERROR: ' + message)
        else:
            sys.exit(1)
    
    
    def parse_args(args):
        try:
            opts, args = getopt.getopt(args, '', ['help', 'cases='])
        except getopt.GetoptError:
            print_usage('Invalid arguments.')
    
        cases = [i for i in range(1, CASES + 1)]
    
        for (opt, val) in opts:
            if opt == '--help':
                print_usage(None)
            elif opt == '--cases':
                try:
                    cases = set(map(int, val.split(',')))
                except ValueError:
                    print_usage('Cases must be comma separated list.')
    
                for i in cases:
                    if i not in range(1, CASES + 1):
                        print_usage('Incorrect cases')
    
        return cases
    
    
    if __name__ == '__main__':
        cases = parse_args(sys.argv[1:])
    
        for i in cases:
            if i == 1:
                case1()
            elif i == 2:
                case2()
            elif i == 3:
                case3()
            elif i == 4:
                case4()
            elif i == 5:
                case5()


\end{lstlisting}

\subsection*{Выводы}
Проведённый эксперимент позволил выявить ключевые закономерности в статистике текстов. Основной результат показывает, что естественные языки обладают выраженной избыточностью: частота совпадений символов (6.48\%) существенно превышает теоретический уровень для случайных данных (3.23\%). 

Тексты из случайных слов демонстрируют промежуточные значения (5.7-5.8\%), сохраняя часть статистических свойств естественного языка. Это подтверждает, что даже при случайной комбинации реальных слов остаются характерные языковые паттерны.

С практической точки зрения, полученные данные подчёркивают важность полного разрушения статистических закономерностей при криптографическом преобразовании текстов. Разница между естественными и случайными последовательностями (более чем в 2 раза) даёт чёткий количественный критерий для оценки криптостойкости алгоритмов.


\end{document}